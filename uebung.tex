\documentclass{uebblatt}

\newcommand{\http}{http:/\kern-.2em/\kern-0.03em}

\begin{document}

\maketitle{1}{Learn You a Haskell for Great Good!}

\begin{aufgabe*}{Aufwärmübungen in GHCI}
  \begin{itemize}
    \item Benutze die vordefinierten Funktionen \haskellinline{fst :: (a, b) -> a} und \haskellinline{snd :: (a, b) -> b}, um das Textzeichen aus \haskellinline{(1, ('a', "foo"))} zu extrahieren.
    \item Sei \haskellinline{xs} eine unsortierte Liste von Zahlen, z.\,B. \haskellinline{let xs = [3, 7, -10, 277, 89, 13, 22, -100, 1]}. Schreibe einen Ausdruck, der den Median (das mittlere Element in einer Sortierung der Liste) von \haskellinline{xs} berechnet.
    \item Was könnte der Ausdruck \haskellinline{(.) . (.)} bewirken? Finde es heraus mit Hilfe von GHCI!
  \end{itemize}
\end{aufgabe*}


\end{document}